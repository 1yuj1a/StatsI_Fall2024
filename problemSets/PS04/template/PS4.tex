\documentclass[12pt,letterpaper]{article}
\usepackage{graphicx,textcomp}
\usepackage{natbib}
\usepackage{setspace}
\usepackage{fullpage}
\usepackage{color}
\usepackage[reqno]{amsmath}
\usepackage{amsthm}
\usepackage{fancyvrb}
\usepackage{amssymb,enumerate}
\usepackage[all]{xy}
\usepackage{endnotes}
\usepackage{lscape}
\newtheorem{com}{Comment}
\usepackage{float}
\usepackage{hyperref}
\newtheorem{lem} {Lemma}
\newtheorem{prop}{Proposition}
\newtheorem{thm}{Theorem}
\newtheorem{defn}{Definition}
\newtheorem{cor}{Corollary}
\newtheorem{obs}{Observation}
\usepackage[compact]{titlesec}
\usepackage{dcolumn}
\usepackage{tikz}
\usetikzlibrary{arrows}
\usepackage{multirow}
\usepackage{xcolor}
\newcolumntype{.}{D{.}{.}{-1}}
\newcolumntype{d}[1]{D{.}{.}{#1}}
\definecolor{light-gray}{gray}{0.65}
\usepackage{url}
\usepackage{listings}
\usepackage{color}

\definecolor{codegreen}{rgb}{0,0.6,0}
\definecolor{codegray}{rgb}{0.5,0.5,0.5}
\definecolor{codepurple}{rgb}{0.58,0,0.82}
\definecolor{backcolour}{rgb}{0.95,0.95,0.92}

\lstdefinestyle{mystyle}{
	backgroundcolor=\color{backcolour},   
	commentstyle=\color{codegreen},
	keywordstyle=\color{magenta},
	numberstyle=\tiny\color{codegray},
	stringstyle=\color{codepurple},
	basicstyle=\footnotesize,
	breakatwhitespace=false,         
	breaklines=true,                 
	captionpos=b,                    
	keepspaces=true,                 
	numbers=left,                    
	numbersep=5pt,                  
	showspaces=false,                
	showstringspaces=false,
	showtabs=false,                  
	tabsize=2
}
\lstset{style=mystyle}
\newcommand{\Sref}[1]{Section~\ref{#1}}
\newtheorem{hyp}{Hypothesis}


\title{Problem Set 4}
\date{Due: November 18, 2024}
\author{Applied Stats/Quant Methods 1 
Jia Lyu-2337006}



\begin{document}
	\maketitle
	\section*{Instructions}
	\begin{itemize}
		\item Please show your work! You may lose points by simply writing in the answer. If the problem requires you to execute commands in \texttt{R}, please include the code you used to get your answers. Please also include the \texttt{.R} file that contains your code. If you are not sure if work needs to be shown for a particular problem, please ask.
		\item Your homework should be submitted electronically on GitHub.
		\item This problem set is due before 23:59 on Monday November 18, 2024. No late assignments will be accepted.
	\end{itemize}



	\vspace{.5cm}
\section*{Question 1: Economics}
\vspace{.25cm}
\noindent 	
In this question, use the \texttt{prestige} dataset in the \texttt{car} library. First, run the following commands:

\begin{verbatim}
install.packages(car)
library(car)
data(Prestige)
help(Prestige)
\end{verbatim} 


\noindent We would like to study whether individuals with higher levels of income have more prestigious jobs. Moreover, we would like to study whether professionals have more prestigious jobs than blue and white collar workers.

\newpage
\begin{enumerate}
	
	\item [(a)]
	Create a new variable \texttt{professional} by recoding the variable \texttt{type} so that professionals are coded as $1$, and blue and white collar workers are coded as $0$ (Hint: \texttt{ifelse}).
		\lstinputlisting[language=R,firstline=10,lastline=14]{PS4_Lyu_Jia.R}
	\begin{verbatim} 
	education income women prestige census
	gov.administrators      13.11  12351 11.16     68.8   1113
	general.managers        12.26  25879  4.02     69.1   1130
	accountants             12.77   9271 15.70     63.4   1171
	purchasing.officers     11.42   8865  9.11     56.8   1175
	chemists                14.62   8403 11.68     73.5   2111
	physicists              15.64  11030  5.13     77.6   2113
	type professional
	gov.administrators  prof            1
	general.managers    prof            1
	accountants             prof            1
	purchasing.officers prof            1
	chemists                   prof            1
	physicists                 prof            1
   \end{verbatim}	

	\item [(b)]
	Run a linear model with \texttt{prestige} as an outcome and \texttt{income}, \texttt{professional}, and the interaction of the two as predictors (Note: this is a continuous $\times$ dummy interaction.)
    \lstinputlisting[language=R, firstline=17, lastline=21]{PS4_Lyu_Jia.R}  
	\begin{verbatim} 
		Call:
		lm(formula = prestige ~ income * professional, data = Prestige)
		
		Residuals:
		Min      1Q  Median      3Q     Max 
		-14.852  -5.332  -1.272   4.658  29.932 
		
		Coefficients:
		Estimate Std. Error t value Pr(>|t|)    
		(Intercept)         21.1422589  2.8044261   7.539 2.93e-11 ***
		income               0.0031709  0.0004993   6.351 7.55e-09 ***
		professional        37.7812800  4.2482744   8.893 4.14e-14 ***
		income:professional -0.0023257  0.0005675  -4.098 8.83e-05 ***
		---
		Signif. codes:  0 ‘***’ 0.001 ‘**’ 0.01 ‘*’ 0.05 ‘.’ 0.1 ‘ ’ 1
		
		Residual standard error: 8.012 on 94 degrees of freedom
		(4 observations deleted due to missingness)
		Multiple R-squared:  0.7872,	Adjusted R-squared:  0.7804 
		F-statistic: 115.9 on 3 and 94 DF,  p-value: < 2.2e-16
	\end{verbatim}	
	
	\item [(c)]
	Write the prediction equation based on the result.
	\begin{verbatim} 
	prestige=21.142+0.003⋅income+37.781⋅professional-0.02⋅income⋅professional
    \end{verbatim}

	\item [(d)]
	Interpret the coefficient for \texttt{income}.
	\begin{verbatim}
	The income coefficient in this regression model is 0.003. For
	non-professional workers (blue and white collar), when 
	professional=0, each $1,000 increase in income is associated with an
	expected increase of 0.003 points in prestige.
    \end{verbatim}	

	\item [(e)]
	Interpret the coefficient for \texttt{professional}.
	\begin{verbatim}
	The coefficient for professional in this regression model is 37.781. This
	means that when professional=1, and income is 0, professionals are
	expected to have a job prestige score 37.781 points higher than
	non-professional workers (blue- and white-collar workers).
	\end{verbatim}	
	\newpage
	\item [(f)]
	What is the effect of a \$1,000 increase in income on prestige score for professional occupations? In other words, we are interested in the marginal effect of income when the variable \texttt{professional} takes the value of $1$. Calculate the change in $\hat{y}$ associated with a \$1,000 increase in income based on your answer for (c).
	\begin{verbatim}
Prestige=21.1423+0.0032⋅Income+37.7813⋅Professional−0.0023⋅Income⋅pofessional
     
When Professional=1:
Prestige=(0.0032−0.0023)⋅Income+58.9236
Prestige=0.0009⋅Income+58.9236
     
For professionals, a $1,000 increase in revenue is expected to result in a
reputation score of 0.9 units.
	\end{verbatim}
	
	\item [(g)]
	What is the effect of changing one's occupations from non-professional to professional when her income is \$6,000? We are interested in the marginal effect of professional jobs when the variable \texttt{income} takes the value of $6,000$. Calculate the change in $\hat{y}$ based on your answer for (c).
	\begin{verbatim}
	Prestige=21.1423+0.0032*Income+37.7813*Professional−0.0023*Income*Professional
	
	When Professional=1:
	prestige = 21.1423+0.0032*6000+37.7812-0.0023*6000 = 64.3235
	
	when profession = 0,
	prestige = 21.1423+0.0032*6000 = 40.3423
	
	Difference Between the Two Prestige Values:
	64.3235 - 40.3423 = 23.9812
	
	Switching from non-professional to professional prestige score increases
 the prestige score by about 23.98 points when earning $6,000
	
\end{verbatim}
	
\end{enumerate}

\newpage

\section*{Question 2: Political Science}
\vspace{.25cm}
\noindent 	Researchers are interested in learning the effect of all of those yard signs on voting preferences.\footnote{Donald P. Green, Jonathan	S. Krasno, Alexander Coppock, Benjamin D. Farrer,	Brandon Lenoir, Joshua N. Zingher. 2016. ``The effects of lawn signs on vote outcomes: Results from four randomized field experiments.'' Electoral Studies 41: 143-150. } Working with a campaign in Fairfax County, Virginia, 131 precincts were randomly divided into a treatment and control group. In 30 precincts, signs were posted around the precinct that read, ``For Sale: Terry McAuliffe. Don't Sellout Virgina on November 5.'' \\

Below is the result of a regression with two variables and a constant.  The dependent variable is the proportion of the vote that went to McAuliff's opponent Ken Cuccinelli. The first variable indicates whether a precinct was randomly assigned to have the sign against McAuliffe posted. The second variable indicates
a precinct that was adjacent to a precinct in the treatment group (since people in those precincts might be exposed to the signs).  \\

\vspace{.5cm}
\begin{table}[!htbp]
	\centering 
	\textbf{Impact of lawn signs on vote share}\\
	\begin{tabular}{@{\extracolsep{5pt}}lccc} 
		\\[-1.8ex] 
		\hline \\[-1.8ex]
		Precinct assigned lawn signs  (n=30)  & 0.042\\
		& (0.016) \\
		Precinct adjacent to lawn signs (n=76) & 0.042 \\
		&  (0.013) \\
		Constant  & 0.302\\
		& (0.011)
		\\
		\hline \\
	\end{tabular}\\
	\footnotesize{\textit{Notes:} $R^2$=0.094, N=131}
\end{table}

\vspace{.5cm}
\begin{enumerate}
	\item [(a)] Use the results from a linear regression to determine whether having these yard signs in a precinct affects vote share (e.g., conduct a hypothesis test with $\alpha = .05$).
		\begin{verbatim}
 Null Hypothesis ( H0 ): The presence of lawn signs does not affect the
 vote share for Ken Cuccinelli (coefficient = 0).
 Alternative Hypothesis ( H1): The presence of lawn signs does affect the
 vote share for Ken Cuccinelli (coefficient ≠ 0).
	 
 Coefficient:  β1=0.042
 Standard error: 0.016
     
 Test Statistic Calculation:
 t=Coefficient/Standard Error =0.042/0.016 =2.625
     
 With 131-2=129 degrees of freedom (since we have two predictors and a
 constant in the model), the critical value for a two-tailed test with α=0.05
 is approximately ±1.96.
     
  Since 2.625 > 1.96,we reject the null hypothesis and conclude that having
  yard signs in a precinct does affect vote share.
     
\end{verbatim}
	
	\item [(b)]  Use the results to determine whether being
	next to precincts with these yard signs affects vote
	share (e.g., conduct a hypothesis test with $\alpha = .05$).
	\begin{verbatim}
	 Null Hypothesis ( H0 ): A candidate's age does not affect the voting
	share (coefficient = 0).
	Alternative Hypothesis ( H1): A candidate's age affects the voting
	share(coefficient ≠ 0).
		
	Coefficient:  β1=0.042
	Standard error: 0.013
		
	Test Statistic Calculation:
	t=Coefficient/Standard Error =0.042/0.013 =3.231
		
	With 131-2=129 degrees of freedom (since we have two predictors
	and a constant in the model), the critical value for a two-tailed test
	with α=0.05 is approximately ±1.96.
		
	Since 3.231 > 1.96,we reject the null hypothesis and conclude that
	being adjacent to a precinct with yard signs does affect vote share.
	   
\end{verbatim}

	\item [(c)] Interpret the coefficient for the constant term substantively.
	\begin{verbatim}
The coefficient for the constant term is 0.302, which represents the
average proportion of the vote that went to McAuliff's opponent Ken
Cuccinelli in precincts where neither the treatment nor adjacency to the
treatment was present. This suggests that, on average, Cuccinelli
received about 30.2% of the vote in these precincts.
    \end{verbatim}	
	
	\item [(d)] Evaluate the model fit for this regression.  What does this	tell us about the importance of yard signs versus other factors that are not modeled?
	\begin{verbatim}
	The R-squared value for the regression model is 0.094, which indicates
 that the model explains only about 9.4% of the variability in vote share.
 This suggests that yard signs and adjacency to precincts with yard signs
 are relatively unimportant factors in determining vote share compared to
 other factors that are not included in the model. It is possible that other
 factors, such as candidate characteristics, campaign spending, or voter
 demographics, may have a larger impact on vote share.

   \end{verbatim}	
\end{enumerate}  


\end{document}
